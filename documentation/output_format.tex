\documentclass[10pt,a4paper]{article}

\usepackage{amsmath}	
\usepackage{fullpage}
\usepackage{setspace}
\usepackage{listings}
\usepackage{xcolor}

\onehalfspacing	

\definecolor{white}{rgb}{1,1,1}
\definecolor{gray}{rgb}{0.75,0.75,0.75}
\definecolor{black}{rgb}{0,0,0}

\lstdefinestyle{mystyle}{
	backgroundcolor=\color{black},
	commentstyle=\color{white},
	keywordstyle=\color{white},
	numberstyle=\scriptsize\color{black},
	stringstyle=\scriptsize\color{white},
	basicstyle=\scriptsize\color{white}\ttfamily,
	breakatwhitespace=false,
	breaklines=true,
	keepspaces=true,
	numbers=left,
	numbersep=5pt,
	showspaces=false,
	showstringspaces=false,
	showtabs=false,
	tabsize=4
}
\lstset{style=mystyle}

% shortcuts for covariant basis vectors
\newcommand{\er}{{\mathbf e}_{\rho}}
\newcommand{\ev}{{\mathbf e}_{\vartheta}}
\newcommand{\ez}{{\mathbf e}_{\zeta}}
\newcommand{\eR}{{\mathbf e}_{R}}
\newcommand{\eP}{{\mathbf e}_{\phi}}
\newcommand{\eZ}{{\mathbf e}_{Z}}

\begin{document}

\section*{Output File Format}

This document explains the output file format for equilibrium solutions computed by DESC.  
The output are text files with the naming convention \verb|output.FILE_NAME|.  
All of the necessary variables to fully define an equilibrium solution are output in the following order: 
grid parameters, fixed-boundary shape, pressure and rotational transform profiles, flux surface shapes, and the boundary function $\lambda$.  
An example output file is included for reference at the end of this document.  
All integers are printed with a total width of 3 characters, and all floating point numbers are printed with a total width of 16 characters including 8 digits after the decimal points.  

\subsection*{Grid Parameters}

The first line of the output file is the heading \verb|! grid|, and the next four lines contain the following information in order: 
\begin{enumerate}
\item \verb|M| (integer): poloidal resolution of the solution
\item \verb|N| (integer): toroidal resolution of the solution
\item \verb|NFP| (integer): number of field periods
\item \verb|Psi| (float): total toroidal magnetic flux through the last closed flux surface, $\psi_a$, in Webers
\end{enumerate}

\subsection*{Fixed-Boundary Shape}

The target shape of the plasma boundary is output for reference in the section of the output file with the heading \verb|! boundary|.  
This is the fixed-boundary input that was used to compute the equilibrium, but the last closed flux surface generally does not match this desired shape exactly.  
The shape of the boundary surface is given as a double Fourier series of the form: 
%
\begin{subequations}
\begin{align}
R^b(\theta,\phi) &= \sum_{n=-N}^{N} \sum_{m=-M}^{M} R^{b}_{mn} \mathcal{G}^{m}_{n}(\theta,\phi) \\
Z^b(\theta,\phi) &= \sum_{n=-N}^{N} \sum_{m=-M}^{M} Z^{b}_{mn} \mathcal{G}^{m}_{n}(\theta,\phi) \\
\label{eq:G}
\mathcal{G}^{m}_{n}(\theta,\phi) &= \begin{cases}
\cos(|m|\theta)\cos(|n|N_{FP}\phi) &\text{for }m\ge0, n\ge0 \\
\cos(|m|\theta)\sin(|n|N_{FP}\phi) &\text{for }m\ge0, n<0 \\
\sin(|m|\theta)\cos(|n|N_{FP}\phi) &\text{for }m<0, n\ge0 \\
\sin(|m|\theta)\sin(|n|N_{FP}\phi) &\text{for }m<0, n<0.
\end{cases}
\end{align}
\end{subequations}
%
The Fourier coefficients $R^{b}_{mn}$ and $Z^{b}_{mn}$ are given by the output variables \verb|bR| and \verb|bZ|, respectively.  
The poloidal and toroidal mode numbers $m$ and $n$ that identify each coefficient are given by the variables \verb|m| and \verb|n| on the same line of the output file as \verb|bR| and \verb|bZ|.  
When stellarator symmetry is enforced, only the $R^{b}_{mn}$ with $mn > 0$ and the $Z^{b}_{mn}$ with $mn < 0$ are nonzero.  
Coefficients with $mn = 0$ are nonzero for $R^{b}_{mn}$ if one of the mode numbers is positive, and nonzero for $Z^{b}_{mn}$ if one of the mode numbers is negative.  
The boundary surface given in the example is equivalent to (using Ptolemy's identities): 
%
\begin{subequations}
\begin{align*}
R^b &= 10 - \cos\theta - 0.3 \cos(\theta-3\phi) \\
Z^b &= \sin\theta - 0.3 \sin(\theta-3\phi).
\end{align*}
\end{subequations}

\subsection*{Pressure \& Rotational Transform Profiles}

The pressure and rotational transform profiles that were used to compute the equilibrium are also output for reference in the section of the output file with the heading \verb|! profiles|.  
These are given as a power series in the flux surface label $\rho \equiv \sqrt{\psi / \psi_a}$ as follows: 
%
\begin{subequations}
\begin{align}
p(\rho) &= \sum_{l=0}^{2M} p_{l} \rho^{l} \\
\iota(\rho) &= \sum_{l=0}^{2M} \iota_{l} \rho^{l}.
\end{align}
\end{subequations}
%
The coefficients $p_{l}$ and $\iota_{l}$ are given by the output variables \verb|cP| and \verb|cI|, respectively.  
The radial order $l$ that identifies each coefficient is given by the variable \verb|l| on the same line of the output file as \verb|cP| and \verb|cI|.  
The profiles given in the example are: 
%
\begin{subequations}
\begin{align*}
p &= 3.4\times10^3 (1-\rho^2)^2 \\
\iota &= 0.5 + 1.5 \rho^2.
\end{align*}
\end{subequations}

\subsection*{Flux Surface Shapes}

The shapes of the flux surfaces are the solution to the equilibrium defined by the fixed-boundary and profile inputs.  
They are given by a Fourier-Zernike basis set with ``fringe'' indexing of the form: 
%
\begin{subequations}
\label{eq:RZbasis}
\begin{align}
R(\rho,\vartheta,\zeta) &= \sum_{n=-N}^{N} \sum_{m=-M}^{M} \sum_{l\in L} R_{lmn} \mathcal{Z}^{m}_{l}(\rho,\vartheta) \mathcal{F}^{n}(\zeta) \\
Z(\rho,\vartheta,\zeta) &= \sum_{n=-N}^{N} \sum_{m=-M}^{M} \sum_{l\in L} Z_{lmn} \mathcal{Z}^{m}_{l}(\rho,\vartheta) \mathcal{F}^{n}(\zeta)
\end{align}
\end{subequations}
%
where $L = |m|, |m|+2, |m|+4, \ldots, 2 M$.  
$\mathcal{F}^{n}(\zeta)$ is the toroidal Fourier series defined as 
%
\begin{equation}
\mathcal{F}^{n}(\zeta) = \begin{cases}
\cos(|n|N_{FP}\zeta) &\text{for }n\ge0 \\
\sin(|n|N_{FP}\zeta) &\text{for }n<0. \\
\end{cases}
\end{equation}
%
$\mathcal{Z}^{m}_{l}(\rho,\vartheta)$ are the Zernike polynomials defined on the unit disc $0\leq\rho\leq1$, $\vartheta\in[0,2\pi)$ as 
%
\begin{equation}
\mathcal{Z}^{m}_{l}(\rho,\vartheta) = \begin{cases}
\mathcal{R}^{|m|}_{l}(\rho) \cos(|m|\vartheta) &\text{for }m\ge0 \\
\mathcal{R}^{|m|}_{l}(\rho) \sin(|m|\vartheta) &\text{for }m<0 \\
\end{cases}
\end{equation}
%
with the radial function 
%
\begin{equation}
\mathcal{R}^{|m|}_{l}(\rho) = \sum^{(l-|m|)/2}_{s=0} \frac{(-1)^s(l-s)!}{s![\frac{1}{2}(l+|m|)-s]![\frac{1}{2}(l-|m|)-s]!} \rho^{l-2s}.
\end{equation}
%
The Fourier-Zernike coefficients $R_{mn}$ and $Z_{mn}$ are given by the variables \verb|cR| and \verb|cZ|, respectively, in the section of the output file with the heading \verb|! surfaces|.  
The indicies $l$, $m$, and $n$ that identify each coefficient are given by the variables \verb|l|, \verb|m|, and \verb|n| on the same line of the output file as \verb|cR| and \verb|cZ|.  
When stellarator symmetry is enforced, only the $R_{mn}$ with $mn > 0$ and the $Z_{mn}$ with $mn < 0$ are nonzero.  
Coefficients with $mn = 0$ are nonzero for $R_{mn}$ if one of the mode numbers is positive, and nonzero for $Z_{mn}$ if one of the mode numbers is negative.  
Lines 45-46 of the example output file give the terms 
%
\begin{subequations}
\begin{align*}
R_{3,1,1} \mathcal{Z}^{1}_{3}(\rho,\vartheta) \mathcal{F}^{1}(\zeta) &= 5.26674681 \times 10^{-2} (3\rho^3-2\rho) \cos(\vartheta) \cos(3\zeta) \\
Z_{2,2,-1} \mathcal{Z}^{2}_{2}(\rho,\vartheta) \mathcal{F}^{-1}(\zeta) &= 5.01543691 \times 10^{-2} \rho^2 \cos(2\vartheta) \sin(3\zeta).
\end{align*}
\end{subequations}

The magnetic field is computed in the straight field-line coordinate system $(\rho,\vartheta,\zeta)$ by 
%
\begin{equation}
\mathbf{B} = B^\vartheta \ev + B^\zeta \ez = \frac{2\psi_a \rho}{2\pi \sqrt{g}} \left( \iota \ev + \ez \right).
\end{equation}
%
The covariant basis vectors are defined as 
\begin{equation}
\er = \begin{bmatrix} \partial_\rho R \\ 0 \\ \partial_\rho Z \end{bmatrix} \hspace{5mm} \ev = \begin{bmatrix} \partial_\vartheta R \\ 0 \\ \partial_\vartheta Z \end{bmatrix} \hspace{5mm} \ez = \begin{bmatrix} \partial_\zeta R \\ R \\ \partial_\zeta Z \end{bmatrix}
\end{equation}
%
and the Jacobian of the coordinate system is $\sqrt{g} = \er\cdot\ev\times\ez$.  
The partial derivatives of $R(\rho,\vartheta,\zeta)$ and $Z(\rho,\vartheta,\zeta)$ are known analytically from (\ref{eq:RZbasis}).  
The components of the magnetic field in the toroidal coordinate system $(R,\phi,Z)$ can be easily computed as $B_i = \mathbf{B} \cdot \mathbf{e}_i$ with $\eR = [1, 0, 0]^T$, $\eP = [0, 1, 0]^T$, and $\eZ = [0, 0, 1]^T$.  

\subsection*{Boundary Function $\lambda$}

The straight field-line angle $\zeta$ is equivalent to the toroidal angle by definition: $\zeta = \phi$.  
The function $\lambda(\theta,\phi)$ relates the straight field-line angle $\vartheta$ to the poloidal angle used to define the boundary surface $\theta$ through the equation $\vartheta = \theta + \lambda(\theta,\phi)$.  
It is used internally to enforce the boundary condition at the last closed flux surface, and is output for reference.  
The function is given as a doubles Fourier series of the form: 
%
\begin{subequations}
\begin{align}
\lambda(\theta,\phi) &= \sum_{n=-N}^{N} \sum_{m=-M}^{M} \lambda_{mn} \mathcal{G}^{m}_{n}(\theta,\phi)
\end{align}
\end{subequations}
%
where $\mathcal{G}^{m}_{n}(\theta,\phi)$ was defined in (\ref{eq:G}).  
The Fourier coefficients $\lambda_{mn}$ are given by the variable \verb|cL| in the section of the output file with the heading \verb|! lambda|.  
Their output format follows the same convention as the boundary coefficients \verb|bR| and \verb|bZ|.  
When stellarator symmetry is enforced, only the coefficients with $mn < 0$ are nonzero.  
Coefficients with $mn = 0$ are nonzero if one of the mode numbers is negative.  

\pagebreak

\section*{Example Output File}

\begin{lstlisting}
! grid
M	=   2
N	=   1
NFP	=   3
Psi	=   1.00000000E+00

! boundary
		m:   0	n:   0	bR =   1.00000000E+01	bZ =   0.00000000E+00
		m:   1	n:   0	bR =  -1.00000000E+00	bZ =   0.00000000E+00
		m:  -1	n:   0	bR =   0.00000000E+00	bZ =   1.00000000E+00
		m:   1	n:   1	bR =  -3.00000000E-01	bZ =   0.00000000E+00
		m:  -1	n:  -1	bR =  -3.00000000E-01	bZ =   0.00000000E+00
		m:  -1	n:   1	bR =   0.00000000E+00	bZ =  -3.00000000E-01
		m:   1	n:  -1	bR =   0.00000000E+00	bZ =   3.00000000E-01

! profiles
l:   0					cP =   3.40000000E+03	cI =   5.00000000E-01
l:   1					cP =   0.00000000E+00	cI =   0.00000000E+00
l:   2					cP =  -6.80000000E+03	cI =   1.50000000E+00
l:   3					cP =   0.00000000E+00	cI =   0.00000000E+00
l:   4					cP =   3.40000000E+03	cI =   0.00000000E+00

! surfaces
l:   0	m:   0	n:  -1	cR =   0.00000000E+00	cZ =  -2.90511418E-03
l:   0	m:   0	n:   0	cR =   9.98274712E+00	cZ =   0.00000000E+00
l:   0	m:   0	n:   1	cR =  -2.90180674E-03	cZ =   0.00000000E+00
l:   1	m:  -1	n:  -1	cR =   2.28896490E-01	cZ =   0.00000000E+00
l:   1	m:  -1	n:   0	cR =   0.00000000E+00	cZ =   9.48092222E-01
l:   1	m:  -1	n:   1	cR =   0.00000000E+00	cZ =  -2.27403979E-01
l:   2	m:   0	n:  -1	cR =   0.00000000E+00	cZ =  -2.41707137E-02
l:   2	m:   0	n:   0	cR =  -1.36531448E-01	cZ =   0.00000000E+00
l:   2	m:   0	n:   1	cR =  -2.41387024E-02	cZ =   0.00000000E+00
l:   1	m:   1	n:  -1	cR =   0.00000000E+00	cZ =   2.24346193E-01
l:   1	m:   1	n:   0	cR =   9.25944834E-01	cZ =   0.00000000E+00
l:   1	m:   1	n:   1	cR =   2.25843613E-01	cZ =   0.00000000E+00
l:   2	m:  -2	n:  -1	cR =   3.34519544E-02	cZ =   0.00000000E+00
l:   2	m:  -2	n:   0	cR =   0.00000000E+00	cZ =   1.58172393E-01
l:   2	m:  -2	n:   1	cR =   0.00000000E+00	cZ =  -5.03483447E-02
l:   3	m:  -1	n:  -1	cR =   4.81316537E-02	cZ =   0.00000000E+00
l:   3	m:  -1	n:   0	cR =   0.00000000E+00	cZ =   3.38024112E-02
l:   3	m:  -1	n:   1	cR =   0.00000000E+00	cZ =  -4.74860303E-02
l:   4	m:   0	n:  -1	cR =   0.00000000E+00	cZ =   2.08609498E-02
l:   4	m:   0	n:   0	cR =   1.33345992E-01	cZ =   0.00000000E+00
l:   4	m:   0	n:   1	cR =   2.07783052E-02	cZ =   0.00000000E+00
l:   3	m:   1	n:  -1	cR =   0.00000000E+00	cZ =   5.20291455E-02
l:   3	m:   1	n:   0	cR =   7.29416666E-02	cZ =   0.00000000E+00
l:   3	m:   1	n:   1	cR =   5.26674681E-02	cZ =   0.00000000E+00
l:   2	m:   2	n:  -1	cR =   0.00000000E+00	cZ =   5.01543691E-02
l:   2	m:   2	n:   0	cR =   1.56388795E-01	cZ =   0.00000000E+00
l:   2	m:   2	n:   1	cR =   3.32590868E-02	cZ =   0.00000000E+00

! lambda
		m:  -2	n:  -1	cL =  -0.00000000E+00
		m:  -2	n:   0	cL =   9.55435813E-03
		m:  -2	n:   1	cL =   2.53333116E-02
		m:  -1	n:  -1	cL =  -0.00000000E+00
		m:  -1	n:   0	cL =   9.91996517E-02
		m:  -1	n:   1	cL =  -1.17417875E-02
		m:   0	n:  -1	cL =   1.75103748E-04
		m:   0	n:   0	cL =  -0.00000000E+00
		m:   0	n:   1	cL =  -0.00000000E+00
		m:   1	n:  -1	cL =   1.16506641E-02
		m:   1	n:   0	cL =  -0.00000000E+00
		m:   1	n:   1	cL =  -0.00000000E+00
\end{lstlisting}

\end{document}